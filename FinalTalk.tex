\documentclass[notheorems,envcountsect,allowframebreaks,xcolor=svgnames,8pt]{beamer}

\usepackage[english]{babel}
\usepackage[utf8]{inputenc}
\usepackage{times}
\usepackage[T1]{fontenc}
\usepackage{bookmark}
\usepackage{amsmath}
\usepackage{bbold}
\usepackage{graphicx}
\usepackage{subfigure}
\usepackage{multimedia}
\usepackage{hyperref}
\usepackage{xcolor}

%%%special symbols%%%

\newcommand{\IS}{\mathbb{S}}
\newcommand{\U}{{\bf U}}
\newcommand{\V}{{\bf V}}
\newcommand{\W}{{\bf W}}
\newcommand{\NN}{\mathbb{N}}
\newcommand{\RR}{\mathbb{R}}
\newcommand{\QQ}{\mathbb{Q}}
\newcommand{\ZZ}{\mathbb{Z}}
\newcommand{\CC}{\mathbb{C}}
\newcommand{\FF}{\mathbb{F}}
\newcommand{\SSS}{\mathfrak{S}}
\newcommand{\dd}{\mathfrak{d}}
\newcommand{\LL}{\mathcal{L}}
\newcommand{\Lie}{\text{Lie}}
\newcommand{\GG}{\mathcal{G}}
\newcommand{\AAA}{\mathcal{A}}
\newcommand{\PPP}{\mathcal{P}}
\newcommand{\res}{\text{res}}
\newcommand{\Mat}{\text{Mat}}
\newcommand{\mm}{\mathfrak{m}}
\newcommand{\RRR}{\mathcal{R}}
\newcommand{\OOO}{\mathcal{O}}
\newcommand{\Conj}{\mbox{Conj}}
\newcommand{\Syl}{\mbox{Syl}}
\newcommand{\Mod}{\mbox{ mod }}
\newcommand{\Aut}{\mbox{Aut}}
\newcommand{\Char}{\mbox{ char }}
\newcommand{\ZZZ}{\mathcal{Z}}
\newcommand{\oo}{$\ddot{\mbox{o}}$}
\newcommand{\Tor}{\mbox{Tor}}
\newcommand{\Hom}{\mbox{Hom}}
\newcommand{\stt}{^{\mbox{\tiny{st}}}}
\newcommand{\ndd}{^{\mbox{\tiny{nd}}}}
\newcommand{\rdd}{^{\mbox{\tiny{rd}}}}
\newcommand{\thh}{^{\mbox{\tiny{th}}}}
\newcommand{\Gal}{\mbox{Gal}}
\newcommand{\ii}{\mathfrak{i}}
\newcommand{\bbox}{\hfill $\blacksquare$}
\newcommand{\ep}{\\ \vspace{.001cm}}
\newcommand{\xx}{\mathbf{x}}
\newcommand{\cnn}{\mathbb{C}^{n \times n}}
\newcommand{\diag}{\mbox{Diag}}
\newcommand{\agl}{AGL(2,\ZZ)}
\newcommand{\gl}{GL(2,\ZZ)}
\newcommand{\ord}{\operatorname{ord}}
\newcommand{\lcm}{\operatorname{lcm}}


\definecolor{cornellred}{rgb}{0.75, 0.01, 0.1}

\mode<presentation> {
  \useoutertheme[footline=authorinstitutetitle]{miniframes}
  \usetheme{Darmstadt}
  \usecolortheme[named=cornellred]{structure}
  \setbeamercovered{transparent}
  \setbeamertemplate{navigation symbols}{}
  \setbeamertemplate{theorem}[ams style]
  \setbeamertemplate{theorems}[numbered]
  \setbeamercolor{frametitle}{fg=structure}
  \setbeamertemplate{frametitle}{
    \raggedleft
    {\bf \insertframetitle}
    \par
  }
}

\newtheorem{theorem}{Theorem}
\newtheorem{definition}{Definition}
\newtheorem{proposition}{Proposition}
\newtheorem{corollary}{Corollary}
\newtheorem{lemma}{Lemma}
\newtheorem{example}{Example}
\newtheorem{examples}{Examples}
\newtheorem{remark}{Remark}
\newtheorem{question}{Question}

\title
{Fusing surface and satellite-derived PM observations to \\ determine the impact of international transport on coastal PM$_{2.5}$ concentrations in the western U.S.}

\author
{Neha Bora \and Tuo Chen \and Dana Cochran \and Kelly Dougan \and Guatam Sabnis \and Chuanping Yu}

\institute[SAMSI]
{Industrial Math/Stat Modeling Workshop \\ Environmental Protection Agency}

\subject{Talks}

%\AtBeginSubsection[] {
%  \begin{frame}<beamer>{Outline}
%  \tableofcontents[currentsection,currentsubsection]
%  \end{frame}
%}

\begin{document}

\begin{frame}
\titlepage
\end{frame}

%\begin{frame}{Outline}
%\tableofcontents
%% You might wish to add the option [pausesections]
%\end{frame}
%------------------------------------------Introduction--------------------------------------------------------
\section*{Introduction}
\begin{frame}{What we plan to talk about}
\begin{itemize}
\item Background

\begin{itemize}
\item What are PM$_{2.5}$ and AOD?
\end{itemize}

\item The Problem
\begin{itemize}
\item Can we use PM measurements to predict past and/or future PM measurements?
\item Can we construct a model to predict PM from AOD measurements?
\item Is there an impact of PM from international sources?
\end{itemize}

\item Data Sources
\begin{itemize}
\item PM sites and data
\item AOD data
\item Other covariates
\end{itemize}

\item Methods

\item Models

\item Experiments

\item Conclusion


\end{itemize}

\end{frame}
%--------------------------------------------------------------------------------------------------------------
%\subsection*{Background}
%\begin{frame}{Background}
%\begin{itemize}
%
%\item The Clean Air Act makes regulations for air pollution levels in the U.S.
%\begin{itemize}
%\item On the west coast, many places in California have violated the amount of pollution allowed by this act.
%\item A possible reason for this is the air transport of pollution from international sources over the Pacific Ocean.
%\end{itemize}
%\item Measurements of land pollution levels on the coast of California can be compared with satellite measurements of pollutants on the Pacific ocean near the land sites.
%
%\end{itemize}
%\end{frame}
%--------------------------------------------------------------------------------------------------------------
\subsection*{Background}
\begin{frame}{Background}
\begin{itemize}

\item Clean Air Act
\begin{itemize}
\item California violates it
\item China to blame?
\end{itemize}
\item Measure pollutants on land, near California coast
\item Compare with satellite measurements
\item informative picture - LA smog
\end{itemize}
\end{frame}
%--------------------------------------------------------------------------------------------------------------
\subsection*{Background}
\begin{frame}{PM and AOD}
\begin{itemize}

\item PM$_{2.5}$ is a particulate matter that is less than $2.5$ micrometers in diameter.
\item add hair picture - PM 2.5 vs PM 10
\item Aerosol Optical Depth (AOD) measures the amount of light from the sun blocked by dust and pollutants.

\end{itemize}
\end{frame}
%------------------------------------------Data Sources--------------------------------------------------------

\section*{Data Sources}
\subsection*{PM}
\begin{frame}{PM}
\begin{itemize}	
\item add picture of sites - all sites on west coast, Hawaii
\item Ground sites of PM$_{2.5}$ measurements, as collected by EPA
\item Data collected either daily, every three days or every six days.
\end{itemize}	
\end{frame}
%--------------------------------------------------------------------------------------------------------------
\subsection*{AOD}
\begin{frame}{AOD}
\begin{itemize}	
\item add picture of strip of data
\item our AOD data only over Pacific Ocean
\item satellites travel around the Earth once every 16 days.
\item AOD data is collected at each location least twice a month.
\end{itemize}		
\end{frame}
%------------------------------------------Methods--------------------------------------------------------
\section*{Methods}
\begin{frame}{Data Cleaning}
\begin{itemize}
\item picked PM sites closest to coast (13 in California, 9 in Hawaii)
\item add map with sites we chose
\item Found closest locations of AOD measurements to these sites
\item Matched data by date of AOD/PM readings

\end{itemize}
\end{frame}
%--------------------------------------------------------------------------------------------------------------
\subsection*{Covariates}
\begin{frame}{Covariates}
\begin{itemize}	
\item wind speed and direction
\item humidity
\item planetary boundary layer height
\item air temperature
\item added values of these measures at each location at each date
\end{itemize}
\end{frame}
%------------------------------------------Models-----------------------------------------------------------
\section*{Models}
\begin{frame}{Models}
\begin{itemize}	
\item PM$_{2.5}$ time series
\item add gif here
\end{itemize}
\end{frame}
%--------------------------------------------------------------------------------------------------------------
\begin{frame}{Models}
\begin{itemize}	
\item Spatial Interpolation of PM$_{2.5}$
\item add gif/image here
\end{itemize}
\end{frame}
%--------------------------------------------------------------------------------------------------------------
\begin{frame}{Models}
\begin{itemize}	
\item Relationships between AVHRR AOD and surface PM
\item Yu - add stuff
\end{itemize}
\end{frame}

%------------------------------------------Experiments-----------------------------------------------------------
\section*{Experiments}
\begin{frame}{}
	
\end{frame}
%-------------------------------------------Conclusions---------------------------------------------------------
\section*{Conclusions}
\begin{frame}{Conclusions}
	
\end{frame}
%--------------------------------------------------------------------------------------------------------------
\subsection*{Future work}
\begin{frame}{Future work}
	
\end{frame}
%--------------------------------------------------------------------------------------------------------------
\begin{frame}{Acknowledgments}
\begin{itemize}
\item faculty mentors
\item EPA
\item We would like to thank the NOAA, specifically Jessica Matthews for helping to explain the satellite data
\item IMSM
\item NSF
\end{itemize}
\end{frame}

%--------------------------------------------------------------------------------------------------------------
\begin{frame}{References}
	\begin{thebibliography}{99}
	\beamertemplatearticlebibitems
	
	\bibitem{liu} 
	
	\bibitem{noaa} National Centers for Environmental Information. National Oceanic and Atmospheric Administration. Department of Commerce, n.d. Web. 23 July 2016. \textit{https://www.ncdc.noaa.gov/cdr/atmospheric/avhrr-aerosol-optical-thickness}.
	
\bibitem{epa} United States Environmental Protection Agency. AirData. EPA, 5 July 2016. Web. 23 July 2016. \textit{https://www3.epa.gov/airdata/}.
	

	\end{thebibliography}
%use this as reference for citations	
%	
%	\bibitem{selfridge} \textcolor{black}{J. Brillhart; D. H. Lehmer; J. L. Selfridge; B. Tuckerman; S. S. Wagstaff, Jr., 
%	Factorizations of $b^n \pm 1$, $b = 2,3,5,6,7,10,11,12$ Up to High Powers, $3\rdd$ edition, Contemporary 
%	Mathematics, Vol. 22, American Math. Soc., Providence, 2002.}
%	
%	\beamertemplatearticlebibitems
%	\bibitem{erdos} \textcolor{black}{P. Erd\H{o}s, On integers of the form $2^k + p$ and some related problems, 
%	\textit{Summa Brasil. Math.} \textbf{2} (1950) 113--123.}
%	
%	\bibitem{mkozekRS} \textcolor{black}{M. Filaseta, C. Finch, M. Kozek, On powers associated with Sierpi\'nski numbers, 
%	Riesel numbers and Polignac's conjecture. \textit{J. Number Theory} \textbf{128} (2008), no. 7, 1916--1940.}
%	
%	\bibitem{marksenior} \textcolor{black}{M. Filaseta; M. Kozek; C. Nicol; J. Selfridge, Composites that remain composite 
%	after changing a digit, \textit{J. Comb. Number Theory} \textbf{2} (2010), no. 1, 25--36 (2011).}
%	\end{thebibliography}
\end{frame}

\end{document}